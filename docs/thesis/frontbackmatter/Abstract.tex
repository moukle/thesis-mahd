%*******************************************************
% Abstract
%*******************************************************
%\renewcommand{\abstractname}{Abstract}
\vspace*{\fill}
\pdfbookmark[1]{Abstract}{Abstract}
\begingroup
\let\clearpage\relax
\let\cleardoublepage\relax
\let\cleardoublepage\relax

\begin{otherlanguage}{american}
	\chapter*{\makebox[\linewidth]{Abstract}}
	Common object detectors need high computing performance.
	They generate lots of \IT{Region Porposals} to identify all objects in a single frame.

	Thus we created a concept only identifying moving objects.
	A simple framework consisting of a background subtractor and Convolutional Neural Network (CNN) is composed.
	First the background subtractor is applied, searching moving objects.
	The interesting regions are then classified by the CNN.

	For evaluation purpose a prototype in scale of 1:100 was built.
	The combination of both components scores a F1 value of $90.2$.
	The average amount of found regions per frame has been reduced to $4.6$, suggesting lower computing requirements.
\end{otherlanguage}

% \newpage
% \cleardoublepage

\vspace{2cm}
\begin{otherlanguage}{ngerman}
	\pdfbookmark[1]{Zusammenfassung}{Zusammenfassung}
	\chapter*{\makebox[\linewidth]{Zusammenfassung}}
	Typische Objekt Detektoren sind rechenintensiv.
	Es werden sehr viele \IT{Region Porposals} generiert, um alle Objekte in starren Bildern zu finden.

	Deshalb wird in dieser Arbeit ein Konzept entwickelt, mit dem nur sich bewegende Objekte erkannt werden.
	Dazu wird ein Background Subtractor mit einem Convolutional Neural Network (CNN) kombiniert.
	Der Background Subtractor sucht zuerst in jedem Bild der Videoaufnahme nach Bewegungen.
	Die gefundenen Regionen werden anschließend vom CNN klassifiziert.

	Zur Evaluation wird ein Prototyp im Maßstab 1:100 nachgebaut.
	Die Kombinationen beider Komponenten ergibt nach Optimierung einen F1 Wert von $90.2$.
	Durch dieses Konzept konnten die durchschnittlich zu klassifizierenden Bereiche auf $4.6$ pro Bild reduziert und der Rechenaufwand dementsprechend gesenkt werden.
\end{otherlanguage}

\endgroup

\vspace*{\fill}