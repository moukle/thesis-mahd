% ****************************************************************************************************
% classicthesis-config.tex
% formerly known as loadpackages.sty, classicthesis-ldpkg.sty, and classicthesis-preamble.sty
% Use it at the beginning of your ClassicThesis.tex, or as a LaTeX Preamble
% in your ClassicThesis.{tex,lyx} with % ****************************************************************************************************
% classicthesis-config.tex
% formerly known as loadpackages.sty, classicthesis-ldpkg.sty, and classicthesis-preamble.sty
% Use it at the beginning of your ClassicThesis.tex, or as a LaTeX Preamble
% in your ClassicThesis.{tex,lyx} with % ****************************************************************************************************
% classicthesis-config.tex
% formerly known as loadpackages.sty, classicthesis-ldpkg.sty, and classicthesis-preamble.sty
% Use it at the beginning of your ClassicThesis.tex, or as a LaTeX Preamble
% in your ClassicThesis.{tex,lyx} with % ****************************************************************************************************
% classicthesis-config.tex
% formerly known as loadpackages.sty, classicthesis-ldpkg.sty, and classicthesis-preamble.sty
% Use it at the beginning of your ClassicThesis.tex, or as a LaTeX Preamble
% in your ClassicThesis.{tex,lyx} with \input{classicthesis-config}
% ****************************************************************************************************
% If you like the classicthesis, then I would appreciate a postcard.
% My address can be found in the file ClassicThesis.pdf. A collection
% of the postcards I received so far is available online at
% http://postcards.miede.de
% ****************************************************************************************************


% ****************************************************************************************************
% 0. Set the encoding of your files. UTF-8 is the only sensible encoding nowadays. If you can't read
% äöüßáéçèê∂åëæƒÏ€ then change the encoding setting in your editor, not the line below. If your editor
% does not support utf8 use another editor!
% ****************************************************************************************************
\PassOptionsToPackage{utf8}{inputenc}
  \usepackage{inputenc}

% ****************************************************************************************************
% 1. Configure classicthesis for your needs here, e.g., remove "drafting" below
% in order to deactivate the time-stamp on the pages
% (see ClassicThesis.pdf for more information):
% ****************************************************************************************************
\PassOptionsToPackage{
  drafting=false,   % print version information on the bottom of the pages
  tocaligned=false, % the left column of the toc will be aligned (no indentation)
  dottedtoc=true,   % page numbers in ToC flushed right
  parts=true,       % use part division
  eulerchapternumbers=true, % use AMS Euler for chapter font (otherwise Palatino)
  linedheaders=false,       % chaper headers will have line above and beneath
  floatperchapter=true,     % numbering per chapter for all floats (i.e., Figure 1.1)
  listings=true,    % load listings package and setup LoL
  subfig=true,      % setup for preloaded subfig package
  eulermath=false,  % use awesome Euler fonts for mathematical formulae (only with pdfLaTeX)
  beramono=true,    % toggle a nice monospaced font (w/ bold)
  minionpro=false   % setup for minion pro font; use minion pro small caps as well (only with pdfLaTeX)
}{classicthesis}


% ****************************************************************************************************
% 2. Personal data and user ad-hoc commands
% ****************************************************************************************************
%\newcommand{\myTitle}{A Classic Thesis Style\xspace}
%\newcommand{\mySubtitle}{An Homage to The Elements of Typographic Style\xspace}
%\newcommand{\myDegree}{Doktor-Ingenieur (Dr.-Ing.)\xspace}
%\newcommand{\myName}{André Miede\xspace}
%\newcommand{\myProf}{Put name here\xspace}
%\newcommand{\myOtherProf}{Put name here\xspace}
%\newcommand{\mySupervisor}{Put name here\xspace}
%\newcommand{\myFaculty}{Put data here\xspace}
%\newcommand{\myDepartment}{Put data here\xspace}
%\newcommand{\myUni}{Put data here\xspace}
%\newcommand{\myLocation}{Saarbrücken\xspace}
%\newcommand{\myTime}{October 2017\xspace}
%\newcommand{\myVersion}{version 4.4}

% ********************************************************************
% Setup, finetuning, and useful commands
% ********************************************************************
\newcounter{dummy} % necessary for correct hyperlinks (to index, bib, etc.)
\newlength{\abcd} % for ab..z string length calculation
\providecommand{\mLyX}{L\kern-.1667em\lower.25em\hbox{Y}\kern-.125emX\@}
\newcommand{\ie}{i.\,e.}
\newcommand{\Ie}{I.\,e.}
\newcommand{\eg}{e.\,g.}
\newcommand{\Eg}{E.\,g.}
% ****************************************************************************************************


% ****************************************************************************************************
% 3. Loading some handy packages
% ****************************************************************************************************
% ********************************************************************
% Packages with options that might require adjustments
% ********************************************************************
%\PassOptionsToPackage{ngerman,american}{babel}   % change this to your language(s), main language last
% Spanish languages need extra options in order to work with this template
%\PassOptionsToPackage{spanish,es-lcroman}{babel}
\usepackage{babel}

\usepackage{csquotes}

\PassOptionsToPackage{%
  %backend=biber,bibencoding=utf8, %instead of bibtex
  backend=bibtex8,bibencoding=ascii,%
  language=auto,%
  style=numeric-comp,%
  %style=alphabetic,%
  %style=authoryear-comp, % Author 1999, 2010
  %bibstyle=authoryear,dashed=false, % dashed: substitute rep. author with ---
  % sorting=nyt, % name, year, title
  sorting=none, % sort by appearance
  maxbibnames=10, % default: 3, et al.
  %backref=true,%
  natbib=true % natbib compatibility mode (\citep and \citet still work)
}{biblatex}
  \usepackage{biblatex}

% \PassOptionsToPackage{fleqn}{amsmath}       % math environments and more by the AMS
  \usepackage{amsmath}

\PassOptionsToPackage{doublespacing}{hdathesis}  % options: abbrev exam big wiwi english master
  \usepackage{hdathesis}

% ********************************************************************
% General useful packages
% ********************************************************************
\PassOptionsToPackage{T1}{fontenc} % T2A for cyrillics
  \usepackage{fontenc}
\usepackage{textcomp} % fix warning with missing font shapes
\usepackage{scrhack} % fix warnings when using KOMA with listings package
\usepackage{xspace} % to get the spacing after macros right
\usepackage{mparhack} % get marginpar right
%\usepackage{fixltx2e} % fixes some LaTeX stuff --> since 2015 in the LaTeX kernel (see below)
% \usepackage[latest]{latexrelease} % emulate newer kernel version if older is detected
\PassOptionsToPackage{printonlyused}{acronym}
  \usepackage{acronym} % nice macros for handling all acronyms in the thesis
  %\renewcommand{\bflabel}[1]{{#1}\hfill} % fix the list of acronyms --> no longer working
  %\renewcommand*{\acsfont}[1]{\textsc{#1}}
  %\renewcommand*{\aclabelfont}[1]{\acsfont{#1}}
  %\def\bflabel#1{{#1\hfill}}
  \def\bflabel#1{{\acsfont{#1}\hfill}}
  % \def\aclabelfont#1{\acsfont{#1}}
% ****************************************************************************************************
% \usepackage{pgfplots} % External TikZ/PGF support (thanks to Andreas Nautsch) %\usetikzlibrary{external}
% \tikzexternalize[mode=list and make, prefix=ext-tikz/]
% \usepackage{tikz}
% ****************************************************************************************************


% ****************************************************************************************************
% 4. Setup floats: tables, (sub)figures, and captions
% ****************************************************************************************************
\usepackage{tabularx} % better tables
  \setlength{\extrarowheight}{3pt} % increase table row height
\newcommand{\tableheadline}[1]{\multicolumn{1}{c}{\spacedlowsmallcaps{#1}}}
\newcommand{\myfloatalign}{\centering} % to be used with each float for alignment
\usepackage{caption}
% Thanks to cgnieder and Claus Lahiri
% http://tex.stackexchange.com/questions/69349/spacedlowsmallcaps-in-caption-label
% [REMOVED DUE TO OTHER PROBLEMS, SEE ISSUE #82]
%\DeclareCaptionLabelFormat{smallcaps}{\bothIfFirst{#1}{~}\MakeTextLowercase{\textsc{#2}}}
%\captionsetup{font=small,labelformat=smallcaps} % format=hang,
\captionsetup{font=small, labelfont=bf} % format=hang,
\usepackage{subfig}
% ****************************************************************************************************


% ****************************************************************************************************
% 5. Setup code listings
% ****************************************************************************************************
\usepackage{listings}
%\lstset{emph={trueIndex,root},emphstyle=\color{BlueViolet}}%\underbar} % for special keywords
\lstset{language=[LaTeX]Tex,%C++,
  morekeywords={PassOptionsToPackage,selectlanguage},
  keywordstyle=\color{RoyalBlue},%\bfseries,
  basicstyle=\small\ttfamily,
  %identifierstyle=\color{NavyBlue},
  commentstyle=\color{Green}\ttfamily,
  stringstyle=\rmfamily,
  numbers=none,%left,%
  numberstyle=\scriptsize,%\tiny
  stepnumber=5,
  numbersep=8pt,
  showstringspaces=false,
  breaklines=true,
  %frameround=ftff,
  %frame=single,
  belowcaptionskip=.75\baselineskip
  %frame=L
}
% ****************************************************************************************************


% ****************************************************************************************************
% 6. PDFLaTeX, hyperreferences, and citation backreferences
% ****************************************************************************************************
% ********************************************************************
% Using PDFLaTeX
% ********************************************************************
\PassOptionsToPackage{hyperfootnotes=false,pdfpagelabels}{hyperref}
  \usepackage{hyperref}  % backref linktocpage pagebackref
%\ifpdf
%\pdfcompresslevel=9
%\pdfadjustspacing=1
%\fi
%\PassOptionsToPackage{pdftex}{graphicx} %%%IVO: driver will be chosen automatically
  \usepackage{graphicx}


% ********************************************************************
% Hyperreferences
% ********************************************************************
\hypersetup{%
  %draft, % hyperref's draft mode, for printing see below
  colorlinks=true, linktocpage=true, pdfstartpage=3, pdfstartview=FitV,%
  % uncomment the following line if you want to have black links (e.g., for printing)
  %colorlinks=false, linktocpage=false, pdfstartpage=3, pdfstartview=FitV, pdfborder={0 0 0},%
  breaklinks=true, pdfpagemode=UseNone, pageanchor=true, pdfpagemode=UseOutlines,%
  plainpages=false, bookmarksnumbered, bookmarksopen=true, bookmarksopenlevel=1,%
  hypertexnames=true, pdfhighlight=/O,%nesting=true,%frenchlinks,%
  urlcolor=baseRed, linkcolor=RoyalBlue, citecolor=RoyalBlue, %pagecolor=RoyalBlue,%
  %urlcolor=Black, linkcolor=Black, citecolor=Black, %pagecolor=Black,%
  pdftitle={\myTitle},%
  pdfauthor={\textcopyright\ \myName, \myUni, \myFaculty},%
  pdfsubject={},%
  pdfkeywords={},%
  pdfcreator={pdfLaTeX},%
  pdfproducer={LaTeX with hyperref and classicthesis}%
}

% ********************************************************************
% Setup autoreferences
% ********************************************************************
% There are some issues regarding autorefnames
% http://www.ureader.de/msg/136221647.aspx
% http://www.tex.ac.uk/cgi-bin/texfaq2html?label=latexwords
% you have to redefine the makros for the
% language you use, e.g., american, ngerman
% (as chosen when loading babel/AtBeginDocument)
% ********************************************************************
\makeatletter
\@ifpackageloaded{babel}%
  {%
    \addto\extrasamerican{%
      \renewcommand*{\figureautorefname}{Figure}%
      \renewcommand*{\tableautorefname}{Table}%
      \renewcommand*{\partautorefname}{Part}%
      \renewcommand*{\chapterautorefname}{Chapter}%
      \renewcommand*{\sectionautorefname}{Section}%
      \renewcommand*{\subsectionautorefname}{Section}%
      \renewcommand*{\subsubsectionautorefname}{Section}%
    }%
    \addto\extrasngerman{%
      \renewcommand*{\paragraphautorefname}{Absatz}%
      \renewcommand*{\subparagraphautorefname}{Unterabsatz}%
      \renewcommand*{\footnoteautorefname}{Fu\"snote}%
      \renewcommand*{\FancyVerbLineautorefname}{Zeile}%
      \renewcommand*{\theoremautorefname}{Theorem}%
      \renewcommand*{\appendixautorefname}{Anhang}%
      \renewcommand*{\equationautorefname}{Gleichung}%
      \renewcommand*{\itemautorefname}{Punkt}%
    }%
      % Fix to getting autorefs for subfigures right (thanks to Belinda Vogt for changing the definition)
      \providecommand{\subfigureautorefname}{\figureautorefname}%
    }{\relax}
\makeatother


% ****************************************************************************************************
% 7. Last calls before the bar closes
% ****************************************************************************************************
% ********************************************************************
% Development Stuff
% ********************************************************************
\listfiles
%\PassOptionsToPackage{l2tabu,orthodox,abort}{nag}
%  \usepackage{nag}
%\PassOptionsToPackage{warning, all}{onlyamsmath}
%  \usepackage{onlyamsmath}

% ********************************************************************
% Last, but not least...
% ********************************************************************
\usepackage{classicthesis}
% ****************************************************************************************************


% ****************************************************************************************************
% 8. Further adjustments (experimental)
% ****************************************************************************************************
% ********************************************************************
% Changing the text area
% ********************************************************************
%\areaset[current]{312pt}{761pt} % 686 (factor 2.2) + 33 head + 42 head \the\footskip
%\setlength{\marginparwidth}{7em}%
%\setlength{\marginparsep}{2em}%

% ********************************************************************
% Using different fonts
% ********************************************************************
% \usepackage[oldstylenums]{kpfonts} % oldstyle notextcomp
% \usepackage[osf]{libertine}
% \usepackage[light,condensed,math]{iwona}
% \renewcommand{\sfdefault}{iwona}
%\usepackage{lmodern} % <-- no osf support :-(
%\usepackage{cfr-lm} %
%\usepackage[urw-garamond]{mathdesign} <-- no osf support :-(
%\usepackage[default,osfigures]{opensans} % scale=0.95
%\usepackage[sfdefault]{FiraSans}
% ********************************************************************
% \usepackage[largesc,osf]{newpxtext}
% Used to fix these:
% https://bitbucket.org/amiede/classicthesis/issues/139/italics-in-pallatino-capitals-chapter
% https://bitbucket.org/amiede/classicthesis/issues/45/problema-testatine-su-classicthesis-style
% ********************************************************************
%\linespread{1.05} % a bit more for Palatino
% ****************************************************************************************************

% CUSTOM STUFF

% FOR CONFUSION MATRIX
\usepackage{multirow}
\usepackage{graphicx}
\newcommand\MyBox[1]{
  \fbox{\lower0.75cm
    \vbox to 1.8cm{\vfil
      \hbox to 1.8cm{\hfil\parbox{1.6cm}{#1}\hfil}
      \vfil}%
  }%
}

\usepackage{tabularcalc}
\usepackage{dirtree}

% FONT
\usepackage[default,light,bold]{sourceserifpro}
\usepackage[T1]{fontenc}

\titleformat{\section}
       {\fontsize{13}{17}\bfseries}{\thesection}{1em}{}
\titleformat{\subsection}
       {\fontsize{11}{17}\bfseries}{\thesubsection}{1em}{}
\titleformat{\subsubsection}
       {\fontsize{10}{17}\bfseries}{\thesubsubsection}{1em}{}

\usepackage{siunitx}
% \sisetup{detect-all}
\DeclareSIUnit\px{px}

\usepackage{cancel}

\usepackage{tikz}
\usepackage{pgfplots}
\pgfplotsset{compat=1.16}

% \usepackage[pdftex,active,tightpage]{preview} % vgg16
\usepackage{float}

\usepackage{enumerate}
\usepackage[inline, shortlabels]{enumitem}

\usepackage[cache=false, newfloat]{minted} % code
\newenvironment{code}{\captionsetup{type=listing}}{}
\SetupFloatingEnvironment{listing}{name=Listing}

\usepackage{blindtext,mdframed}

\usepackage{booktabs}
\newcommand{\tablestyle}[2]{\setlength{\tabcolsep}{#1}\renewcommand{\arraystretch}{#2}\centering\footnotesize}
\newlength\savewidth\newcommand\shline{\noalign{\global\savewidth\arrayrulewidth 
    \global\arrayrulewidth 1pt}\hline\noalign{\global\arrayrulewidth\savewidth}}   

\usepackage{makecell}
\usepackage{cellspace}
% \addparagraphcolumntypes{X}
\setlength\cellspacetoplimit{10pt}
\setlength\cellspacebottomlimit{10pt}

\usepackage[ngerman]{datetime}
\newdateformat{gerdate}{\THEDAY{ten }\monthnamengerman[\THEMONTH], \THEYEAR}

\newcommand{\TT}[1]{\texttt{#1}}
\newcommand{\BF}[1]{\textbf{#1}}
\newcommand{\IT}[1]{\textit{#1}}
\newcommand{\SC}[1]{\textsc{#1}}
\def\TODO{\textcolor{red}{\emph{TODO: }}}

\usepackage[ruled,vlined]{algorithm2e}
\usepackage[noend]{algpseudocode}

\usepackage{glossaries}
\renewcommand*\glspostdescription{\dotfill\nolinebreak}
\renewcommand*{\glstextformat}[1]{\slshape\color{RoyalBlue}{#1}}
\makeglossaries

\newglossaryentry{supervised}
{
    name=Supervised Learning,
    description={Das Lernen durch \textit{Pre-Labeled} Eingaben. Jedes Trainingsbeispiel besteht aus Eingabewerten $X$ und Zielwert $y$. Das Ziel ist es, den Klassifikationserror eines Modells zu reduzieren, sodass das Modell die korrekte Vorhersage $y$ trifft.}
}

\newglossaryentry{unsupervised}
{
    name=Unsupervised Learning,
    description={Unterscheidet sich vom \gls{supervised} dadurch, dass das Trainingsbeispiel keine Labels enthält. Dabei sollen selbstständig Muster in den Daten erkannt werden.}
}

\newglossaryentry{sliding}
{
    name=Sliding Window,
    description={In der Computer Vision ist damit ein \textit{Fenster} gemeint, welches über das gesamte Bild geschoben wird. Oft werden Fenster mit verschiedenen Größen und Seitenverhältnissen verwendet.}
}

\newglossaryentry{clustering}
{
    name=Clustering,
    description={Ist eine Methode des \gls{unsupervised}. Dabei werden Datenpunkte in $n$ Gruppen geteilt, sodass ähnliche Datenpunkte der gleichen Gruppe zugeordnet werden.}
}

\newglossaryentry{hysterese}
{
    name=Hysterese,
    description={Ist ein Systemverhalten, bei dem die Ausgabe abhängig von vorherigen Ausgaben ist. So können gleiche Eingaben zu unterschiedlichen Ausgaben führen.}
}

\setlength\parindent{0pt}
% ****************************************************************************************************
% If you like the classicthesis, then I would appreciate a postcard.
% My address can be found in the file ClassicThesis.pdf. A collection
% of the postcards I received so far is available online at
% http://postcards.miede.de
% ****************************************************************************************************


% ****************************************************************************************************
% 0. Set the encoding of your files. UTF-8 is the only sensible encoding nowadays. If you can't read
% äöüßáéçèê∂åëæƒÏ€ then change the encoding setting in your editor, not the line below. If your editor
% does not support utf8 use another editor!
% ****************************************************************************************************
\PassOptionsToPackage{utf8}{inputenc}
  \usepackage{inputenc}

% ****************************************************************************************************
% 1. Configure classicthesis for your needs here, e.g., remove "drafting" below
% in order to deactivate the time-stamp on the pages
% (see ClassicThesis.pdf for more information):
% ****************************************************************************************************
\PassOptionsToPackage{
  drafting=false,   % print version information on the bottom of the pages
  tocaligned=false, % the left column of the toc will be aligned (no indentation)
  dottedtoc=true,   % page numbers in ToC flushed right
  parts=true,       % use part division
  eulerchapternumbers=true, % use AMS Euler for chapter font (otherwise Palatino)
  linedheaders=false,       % chaper headers will have line above and beneath
  floatperchapter=true,     % numbering per chapter for all floats (i.e., Figure 1.1)
  listings=true,    % load listings package and setup LoL
  subfig=true,      % setup for preloaded subfig package
  eulermath=false,  % use awesome Euler fonts for mathematical formulae (only with pdfLaTeX)
  beramono=true,    % toggle a nice monospaced font (w/ bold)
  minionpro=false   % setup for minion pro font; use minion pro small caps as well (only with pdfLaTeX)
}{classicthesis}


% ****************************************************************************************************
% 2. Personal data and user ad-hoc commands
% ****************************************************************************************************
%\newcommand{\myTitle}{A Classic Thesis Style\xspace}
%\newcommand{\mySubtitle}{An Homage to The Elements of Typographic Style\xspace}
%\newcommand{\myDegree}{Doktor-Ingenieur (Dr.-Ing.)\xspace}
%\newcommand{\myName}{André Miede\xspace}
%\newcommand{\myProf}{Put name here\xspace}
%\newcommand{\myOtherProf}{Put name here\xspace}
%\newcommand{\mySupervisor}{Put name here\xspace}
%\newcommand{\myFaculty}{Put data here\xspace}
%\newcommand{\myDepartment}{Put data here\xspace}
%\newcommand{\myUni}{Put data here\xspace}
%\newcommand{\myLocation}{Saarbrücken\xspace}
%\newcommand{\myTime}{October 2017\xspace}
%\newcommand{\myVersion}{version 4.4}

% ********************************************************************
% Setup, finetuning, and useful commands
% ********************************************************************
\newcounter{dummy} % necessary for correct hyperlinks (to index, bib, etc.)
\newlength{\abcd} % for ab..z string length calculation
\providecommand{\mLyX}{L\kern-.1667em\lower.25em\hbox{Y}\kern-.125emX\@}
\newcommand{\ie}{i.\,e.}
\newcommand{\Ie}{I.\,e.}
\newcommand{\eg}{e.\,g.}
\newcommand{\Eg}{E.\,g.}
% ****************************************************************************************************


% ****************************************************************************************************
% 3. Loading some handy packages
% ****************************************************************************************************
% ********************************************************************
% Packages with options that might require adjustments
% ********************************************************************
%\PassOptionsToPackage{ngerman,american}{babel}   % change this to your language(s), main language last
% Spanish languages need extra options in order to work with this template
%\PassOptionsToPackage{spanish,es-lcroman}{babel}
\usepackage{babel}

\usepackage{csquotes}

\PassOptionsToPackage{%
  %backend=biber,bibencoding=utf8, %instead of bibtex
  backend=bibtex8,bibencoding=ascii,%
  language=auto,%
  style=numeric-comp,%
  %style=alphabetic,%
  %style=authoryear-comp, % Author 1999, 2010
  %bibstyle=authoryear,dashed=false, % dashed: substitute rep. author with ---
  % sorting=nyt, % name, year, title
  sorting=none, % sort by appearance
  maxbibnames=10, % default: 3, et al.
  %backref=true,%
  natbib=true % natbib compatibility mode (\citep and \citet still work)
}{biblatex}
  \usepackage{biblatex}

% \PassOptionsToPackage{fleqn}{amsmath}       % math environments and more by the AMS
  \usepackage{amsmath}

\PassOptionsToPackage{doublespacing}{hdathesis}  % options: abbrev exam big wiwi english master
  \usepackage{hdathesis}

% ********************************************************************
% General useful packages
% ********************************************************************
\PassOptionsToPackage{T1}{fontenc} % T2A for cyrillics
  \usepackage{fontenc}
\usepackage{textcomp} % fix warning with missing font shapes
\usepackage{scrhack} % fix warnings when using KOMA with listings package
\usepackage{xspace} % to get the spacing after macros right
\usepackage{mparhack} % get marginpar right
%\usepackage{fixltx2e} % fixes some LaTeX stuff --> since 2015 in the LaTeX kernel (see below)
% \usepackage[latest]{latexrelease} % emulate newer kernel version if older is detected
\PassOptionsToPackage{printonlyused}{acronym}
  \usepackage{acronym} % nice macros for handling all acronyms in the thesis
  %\renewcommand{\bflabel}[1]{{#1}\hfill} % fix the list of acronyms --> no longer working
  %\renewcommand*{\acsfont}[1]{\textsc{#1}}
  %\renewcommand*{\aclabelfont}[1]{\acsfont{#1}}
  %\def\bflabel#1{{#1\hfill}}
  \def\bflabel#1{{\acsfont{#1}\hfill}}
  % \def\aclabelfont#1{\acsfont{#1}}
% ****************************************************************************************************
% \usepackage{pgfplots} % External TikZ/PGF support (thanks to Andreas Nautsch) %\usetikzlibrary{external}
% \tikzexternalize[mode=list and make, prefix=ext-tikz/]
% \usepackage{tikz}
% ****************************************************************************************************


% ****************************************************************************************************
% 4. Setup floats: tables, (sub)figures, and captions
% ****************************************************************************************************
\usepackage{tabularx} % better tables
  \setlength{\extrarowheight}{3pt} % increase table row height
\newcommand{\tableheadline}[1]{\multicolumn{1}{c}{\spacedlowsmallcaps{#1}}}
\newcommand{\myfloatalign}{\centering} % to be used with each float for alignment
\usepackage{caption}
% Thanks to cgnieder and Claus Lahiri
% http://tex.stackexchange.com/questions/69349/spacedlowsmallcaps-in-caption-label
% [REMOVED DUE TO OTHER PROBLEMS, SEE ISSUE #82]
%\DeclareCaptionLabelFormat{smallcaps}{\bothIfFirst{#1}{~}\MakeTextLowercase{\textsc{#2}}}
%\captionsetup{font=small,labelformat=smallcaps} % format=hang,
\captionsetup{font=small, labelfont=bf} % format=hang,
\usepackage{subfig}
% ****************************************************************************************************


% ****************************************************************************************************
% 5. Setup code listings
% ****************************************************************************************************
\usepackage{listings}
%\lstset{emph={trueIndex,root},emphstyle=\color{BlueViolet}}%\underbar} % for special keywords
\lstset{language=[LaTeX]Tex,%C++,
  morekeywords={PassOptionsToPackage,selectlanguage},
  keywordstyle=\color{RoyalBlue},%\bfseries,
  basicstyle=\small\ttfamily,
  %identifierstyle=\color{NavyBlue},
  commentstyle=\color{Green}\ttfamily,
  stringstyle=\rmfamily,
  numbers=none,%left,%
  numberstyle=\scriptsize,%\tiny
  stepnumber=5,
  numbersep=8pt,
  showstringspaces=false,
  breaklines=true,
  %frameround=ftff,
  %frame=single,
  belowcaptionskip=.75\baselineskip
  %frame=L
}
% ****************************************************************************************************


% ****************************************************************************************************
% 6. PDFLaTeX, hyperreferences, and citation backreferences
% ****************************************************************************************************
% ********************************************************************
% Using PDFLaTeX
% ********************************************************************
\PassOptionsToPackage{hyperfootnotes=false,pdfpagelabels}{hyperref}
  \usepackage{hyperref}  % backref linktocpage pagebackref
%\ifpdf
%\pdfcompresslevel=9
%\pdfadjustspacing=1
%\fi
%\PassOptionsToPackage{pdftex}{graphicx} %%%IVO: driver will be chosen automatically
  \usepackage{graphicx}


% ********************************************************************
% Hyperreferences
% ********************************************************************
\hypersetup{%
  %draft, % hyperref's draft mode, for printing see below
  colorlinks=true, linktocpage=true, pdfstartpage=3, pdfstartview=FitV,%
  % uncomment the following line if you want to have black links (e.g., for printing)
  %colorlinks=false, linktocpage=false, pdfstartpage=3, pdfstartview=FitV, pdfborder={0 0 0},%
  breaklinks=true, pdfpagemode=UseNone, pageanchor=true, pdfpagemode=UseOutlines,%
  plainpages=false, bookmarksnumbered, bookmarksopen=true, bookmarksopenlevel=1,%
  hypertexnames=true, pdfhighlight=/O,%nesting=true,%frenchlinks,%
  urlcolor=baseRed, linkcolor=RoyalBlue, citecolor=RoyalBlue, %pagecolor=RoyalBlue,%
  %urlcolor=Black, linkcolor=Black, citecolor=Black, %pagecolor=Black,%
  pdftitle={\myTitle},%
  pdfauthor={\textcopyright\ \myName, \myUni, \myFaculty},%
  pdfsubject={},%
  pdfkeywords={},%
  pdfcreator={pdfLaTeX},%
  pdfproducer={LaTeX with hyperref and classicthesis}%
}

% ********************************************************************
% Setup autoreferences
% ********************************************************************
% There are some issues regarding autorefnames
% http://www.ureader.de/msg/136221647.aspx
% http://www.tex.ac.uk/cgi-bin/texfaq2html?label=latexwords
% you have to redefine the makros for the
% language you use, e.g., american, ngerman
% (as chosen when loading babel/AtBeginDocument)
% ********************************************************************
\makeatletter
\@ifpackageloaded{babel}%
  {%
    \addto\extrasamerican{%
      \renewcommand*{\figureautorefname}{Figure}%
      \renewcommand*{\tableautorefname}{Table}%
      \renewcommand*{\partautorefname}{Part}%
      \renewcommand*{\chapterautorefname}{Chapter}%
      \renewcommand*{\sectionautorefname}{Section}%
      \renewcommand*{\subsectionautorefname}{Section}%
      \renewcommand*{\subsubsectionautorefname}{Section}%
    }%
    \addto\extrasngerman{%
      \renewcommand*{\paragraphautorefname}{Absatz}%
      \renewcommand*{\subparagraphautorefname}{Unterabsatz}%
      \renewcommand*{\footnoteautorefname}{Fu\"snote}%
      \renewcommand*{\FancyVerbLineautorefname}{Zeile}%
      \renewcommand*{\theoremautorefname}{Theorem}%
      \renewcommand*{\appendixautorefname}{Anhang}%
      \renewcommand*{\equationautorefname}{Gleichung}%
      \renewcommand*{\itemautorefname}{Punkt}%
    }%
      % Fix to getting autorefs for subfigures right (thanks to Belinda Vogt for changing the definition)
      \providecommand{\subfigureautorefname}{\figureautorefname}%
    }{\relax}
\makeatother


% ****************************************************************************************************
% 7. Last calls before the bar closes
% ****************************************************************************************************
% ********************************************************************
% Development Stuff
% ********************************************************************
\listfiles
%\PassOptionsToPackage{l2tabu,orthodox,abort}{nag}
%  \usepackage{nag}
%\PassOptionsToPackage{warning, all}{onlyamsmath}
%  \usepackage{onlyamsmath}

% ********************************************************************
% Last, but not least...
% ********************************************************************
\usepackage{classicthesis}
% ****************************************************************************************************


% ****************************************************************************************************
% 8. Further adjustments (experimental)
% ****************************************************************************************************
% ********************************************************************
% Changing the text area
% ********************************************************************
%\areaset[current]{312pt}{761pt} % 686 (factor 2.2) + 33 head + 42 head \the\footskip
%\setlength{\marginparwidth}{7em}%
%\setlength{\marginparsep}{2em}%

% ********************************************************************
% Using different fonts
% ********************************************************************
% \usepackage[oldstylenums]{kpfonts} % oldstyle notextcomp
% \usepackage[osf]{libertine}
% \usepackage[light,condensed,math]{iwona}
% \renewcommand{\sfdefault}{iwona}
%\usepackage{lmodern} % <-- no osf support :-(
%\usepackage{cfr-lm} %
%\usepackage[urw-garamond]{mathdesign} <-- no osf support :-(
%\usepackage[default,osfigures]{opensans} % scale=0.95
%\usepackage[sfdefault]{FiraSans}
% ********************************************************************
% \usepackage[largesc,osf]{newpxtext}
% Used to fix these:
% https://bitbucket.org/amiede/classicthesis/issues/139/italics-in-pallatino-capitals-chapter
% https://bitbucket.org/amiede/classicthesis/issues/45/problema-testatine-su-classicthesis-style
% ********************************************************************
%\linespread{1.05} % a bit more for Palatino
% ****************************************************************************************************

% CUSTOM STUFF

% FOR CONFUSION MATRIX
\usepackage{multirow}
\usepackage{graphicx}
\newcommand\MyBox[1]{
  \fbox{\lower0.75cm
    \vbox to 1.8cm{\vfil
      \hbox to 1.8cm{\hfil\parbox{1.6cm}{#1}\hfil}
      \vfil}%
  }%
}

\usepackage{tabularcalc}
\usepackage{dirtree}

% FONT
\usepackage[default,light,bold]{sourceserifpro}
\usepackage[T1]{fontenc}

\titleformat{\section}
       {\fontsize{13}{17}\bfseries}{\thesection}{1em}{}
\titleformat{\subsection}
       {\fontsize{11}{17}\bfseries}{\thesubsection}{1em}{}
\titleformat{\subsubsection}
       {\fontsize{10}{17}\bfseries}{\thesubsubsection}{1em}{}

\usepackage{siunitx}
% \sisetup{detect-all}
\DeclareSIUnit\px{px}

\usepackage{cancel}

\usepackage{tikz}
\usepackage{pgfplots}
\pgfplotsset{compat=1.16}

% \usepackage[pdftex,active,tightpage]{preview} % vgg16
\usepackage{float}

\usepackage{enumerate}
\usepackage[inline, shortlabels]{enumitem}

\usepackage[cache=false, newfloat]{minted} % code
\newenvironment{code}{\captionsetup{type=listing}}{}
\SetupFloatingEnvironment{listing}{name=Listing}

\usepackage{blindtext,mdframed}

\usepackage{booktabs}
\newcommand{\tablestyle}[2]{\setlength{\tabcolsep}{#1}\renewcommand{\arraystretch}{#2}\centering\footnotesize}
\newlength\savewidth\newcommand\shline{\noalign{\global\savewidth\arrayrulewidth 
    \global\arrayrulewidth 1pt}\hline\noalign{\global\arrayrulewidth\savewidth}}   

\usepackage{makecell}
\usepackage{cellspace}
% \addparagraphcolumntypes{X}
\setlength\cellspacetoplimit{10pt}
\setlength\cellspacebottomlimit{10pt}

\usepackage[ngerman]{datetime}
\newdateformat{gerdate}{\THEDAY{ten }\monthnamengerman[\THEMONTH], \THEYEAR}

\newcommand{\TT}[1]{\texttt{#1}}
\newcommand{\BF}[1]{\textbf{#1}}
\newcommand{\IT}[1]{\textit{#1}}
\newcommand{\SC}[1]{\textsc{#1}}
\def\TODO{\textcolor{red}{\emph{TODO: }}}

\usepackage[ruled,vlined]{algorithm2e}
\usepackage[noend]{algpseudocode}

\usepackage{glossaries}
\renewcommand*\glspostdescription{\dotfill\nolinebreak}
\renewcommand*{\glstextformat}[1]{\slshape\color{RoyalBlue}{#1}}
\makeglossaries

\newglossaryentry{supervised}
{
    name=Supervised Learning,
    description={Das Lernen durch \textit{Pre-Labeled} Eingaben. Jedes Trainingsbeispiel besteht aus Eingabewerten $X$ und Zielwert $y$. Das Ziel ist es, den Klassifikationserror eines Modells zu reduzieren, sodass das Modell die korrekte Vorhersage $y$ trifft.}
}

\newglossaryentry{unsupervised}
{
    name=Unsupervised Learning,
    description={Unterscheidet sich vom \gls{supervised} dadurch, dass das Trainingsbeispiel keine Labels enthält. Dabei sollen selbstständig Muster in den Daten erkannt werden.}
}

\newglossaryentry{sliding}
{
    name=Sliding Window,
    description={In der Computer Vision ist damit ein \textit{Fenster} gemeint, welches über das gesamte Bild geschoben wird. Oft werden Fenster mit verschiedenen Größen und Seitenverhältnissen verwendet.}
}

\newglossaryentry{clustering}
{
    name=Clustering,
    description={Ist eine Methode des \gls{unsupervised}. Dabei werden Datenpunkte in $n$ Gruppen geteilt, sodass ähnliche Datenpunkte der gleichen Gruppe zugeordnet werden.}
}

\newglossaryentry{hysterese}
{
    name=Hysterese,
    description={Ist ein Systemverhalten, bei dem die Ausgabe abhängig von vorherigen Ausgaben ist. So können gleiche Eingaben zu unterschiedlichen Ausgaben führen.}
}

\setlength\parindent{0pt}
% ****************************************************************************************************
% If you like the classicthesis, then I would appreciate a postcard.
% My address can be found in the file ClassicThesis.pdf. A collection
% of the postcards I received so far is available online at
% http://postcards.miede.de
% ****************************************************************************************************


% ****************************************************************************************************
% 0. Set the encoding of your files. UTF-8 is the only sensible encoding nowadays. If you can't read
% äöüßáéçèê∂åëæƒÏ€ then change the encoding setting in your editor, not the line below. If your editor
% does not support utf8 use another editor!
% ****************************************************************************************************
\PassOptionsToPackage{utf8}{inputenc}
  \usepackage{inputenc}

% ****************************************************************************************************
% 1. Configure classicthesis for your needs here, e.g., remove "drafting" below
% in order to deactivate the time-stamp on the pages
% (see ClassicThesis.pdf for more information):
% ****************************************************************************************************
\PassOptionsToPackage{
  drafting=false,   % print version information on the bottom of the pages
  tocaligned=false, % the left column of the toc will be aligned (no indentation)
  dottedtoc=true,   % page numbers in ToC flushed right
  parts=true,       % use part division
  eulerchapternumbers=true, % use AMS Euler for chapter font (otherwise Palatino)
  linedheaders=false,       % chaper headers will have line above and beneath
  floatperchapter=true,     % numbering per chapter for all floats (i.e., Figure 1.1)
  listings=true,    % load listings package and setup LoL
  subfig=true,      % setup for preloaded subfig package
  eulermath=false,  % use awesome Euler fonts for mathematical formulae (only with pdfLaTeX)
  beramono=true,    % toggle a nice monospaced font (w/ bold)
  minionpro=false   % setup for minion pro font; use minion pro small caps as well (only with pdfLaTeX)
}{classicthesis}


% ****************************************************************************************************
% 2. Personal data and user ad-hoc commands
% ****************************************************************************************************
%\newcommand{\myTitle}{A Classic Thesis Style\xspace}
%\newcommand{\mySubtitle}{An Homage to The Elements of Typographic Style\xspace}
%\newcommand{\myDegree}{Doktor-Ingenieur (Dr.-Ing.)\xspace}
%\newcommand{\myName}{André Miede\xspace}
%\newcommand{\myProf}{Put name here\xspace}
%\newcommand{\myOtherProf}{Put name here\xspace}
%\newcommand{\mySupervisor}{Put name here\xspace}
%\newcommand{\myFaculty}{Put data here\xspace}
%\newcommand{\myDepartment}{Put data here\xspace}
%\newcommand{\myUni}{Put data here\xspace}
%\newcommand{\myLocation}{Saarbrücken\xspace}
%\newcommand{\myTime}{October 2017\xspace}
%\newcommand{\myVersion}{version 4.4}

% ********************************************************************
% Setup, finetuning, and useful commands
% ********************************************************************
\newcounter{dummy} % necessary for correct hyperlinks (to index, bib, etc.)
\newlength{\abcd} % for ab..z string length calculation
\providecommand{\mLyX}{L\kern-.1667em\lower.25em\hbox{Y}\kern-.125emX\@}
\newcommand{\ie}{i.\,e.}
\newcommand{\Ie}{I.\,e.}
\newcommand{\eg}{e.\,g.}
\newcommand{\Eg}{E.\,g.}
% ****************************************************************************************************


% ****************************************************************************************************
% 3. Loading some handy packages
% ****************************************************************************************************
% ********************************************************************
% Packages with options that might require adjustments
% ********************************************************************
%\PassOptionsToPackage{ngerman,american}{babel}   % change this to your language(s), main language last
% Spanish languages need extra options in order to work with this template
%\PassOptionsToPackage{spanish,es-lcroman}{babel}
\usepackage{babel}

\usepackage{csquotes}

\PassOptionsToPackage{%
  %backend=biber,bibencoding=utf8, %instead of bibtex
  backend=bibtex8,bibencoding=ascii,%
  language=auto,%
  style=numeric-comp,%
  %style=alphabetic,%
  %style=authoryear-comp, % Author 1999, 2010
  %bibstyle=authoryear,dashed=false, % dashed: substitute rep. author with ---
  % sorting=nyt, % name, year, title
  sorting=none, % sort by appearance
  maxbibnames=10, % default: 3, et al.
  %backref=true,%
  natbib=true % natbib compatibility mode (\citep and \citet still work)
}{biblatex}
  \usepackage{biblatex}

% \PassOptionsToPackage{fleqn}{amsmath}       % math environments and more by the AMS
  \usepackage{amsmath}

\PassOptionsToPackage{doublespacing}{hdathesis}  % options: abbrev exam big wiwi english master
  \usepackage{hdathesis}

% ********************************************************************
% General useful packages
% ********************************************************************
\PassOptionsToPackage{T1}{fontenc} % T2A for cyrillics
  \usepackage{fontenc}
\usepackage{textcomp} % fix warning with missing font shapes
\usepackage{scrhack} % fix warnings when using KOMA with listings package
\usepackage{xspace} % to get the spacing after macros right
\usepackage{mparhack} % get marginpar right
%\usepackage{fixltx2e} % fixes some LaTeX stuff --> since 2015 in the LaTeX kernel (see below)
% \usepackage[latest]{latexrelease} % emulate newer kernel version if older is detected
\PassOptionsToPackage{printonlyused}{acronym}
  \usepackage{acronym} % nice macros for handling all acronyms in the thesis
  %\renewcommand{\bflabel}[1]{{#1}\hfill} % fix the list of acronyms --> no longer working
  %\renewcommand*{\acsfont}[1]{\textsc{#1}}
  %\renewcommand*{\aclabelfont}[1]{\acsfont{#1}}
  %\def\bflabel#1{{#1\hfill}}
  \def\bflabel#1{{\acsfont{#1}\hfill}}
  % \def\aclabelfont#1{\acsfont{#1}}
% ****************************************************************************************************
% \usepackage{pgfplots} % External TikZ/PGF support (thanks to Andreas Nautsch) %\usetikzlibrary{external}
% \tikzexternalize[mode=list and make, prefix=ext-tikz/]
% \usepackage{tikz}
% ****************************************************************************************************


% ****************************************************************************************************
% 4. Setup floats: tables, (sub)figures, and captions
% ****************************************************************************************************
\usepackage{tabularx} % better tables
  \setlength{\extrarowheight}{3pt} % increase table row height
\newcommand{\tableheadline}[1]{\multicolumn{1}{c}{\spacedlowsmallcaps{#1}}}
\newcommand{\myfloatalign}{\centering} % to be used with each float for alignment
\usepackage{caption}
% Thanks to cgnieder and Claus Lahiri
% http://tex.stackexchange.com/questions/69349/spacedlowsmallcaps-in-caption-label
% [REMOVED DUE TO OTHER PROBLEMS, SEE ISSUE #82]
%\DeclareCaptionLabelFormat{smallcaps}{\bothIfFirst{#1}{~}\MakeTextLowercase{\textsc{#2}}}
%\captionsetup{font=small,labelformat=smallcaps} % format=hang,
\captionsetup{font=small, labelfont=bf} % format=hang,
\usepackage{subfig}
% ****************************************************************************************************


% ****************************************************************************************************
% 5. Setup code listings
% ****************************************************************************************************
\usepackage{listings}
%\lstset{emph={trueIndex,root},emphstyle=\color{BlueViolet}}%\underbar} % for special keywords
\lstset{language=[LaTeX]Tex,%C++,
  morekeywords={PassOptionsToPackage,selectlanguage},
  keywordstyle=\color{RoyalBlue},%\bfseries,
  basicstyle=\small\ttfamily,
  %identifierstyle=\color{NavyBlue},
  commentstyle=\color{Green}\ttfamily,
  stringstyle=\rmfamily,
  numbers=none,%left,%
  numberstyle=\scriptsize,%\tiny
  stepnumber=5,
  numbersep=8pt,
  showstringspaces=false,
  breaklines=true,
  %frameround=ftff,
  %frame=single,
  belowcaptionskip=.75\baselineskip
  %frame=L
}
% ****************************************************************************************************


% ****************************************************************************************************
% 6. PDFLaTeX, hyperreferences, and citation backreferences
% ****************************************************************************************************
% ********************************************************************
% Using PDFLaTeX
% ********************************************************************
\PassOptionsToPackage{hyperfootnotes=false,pdfpagelabels}{hyperref}
  \usepackage{hyperref}  % backref linktocpage pagebackref
%\ifpdf
%\pdfcompresslevel=9
%\pdfadjustspacing=1
%\fi
%\PassOptionsToPackage{pdftex}{graphicx} %%%IVO: driver will be chosen automatically
  \usepackage{graphicx}


% ********************************************************************
% Hyperreferences
% ********************************************************************
\hypersetup{%
  %draft, % hyperref's draft mode, for printing see below
  colorlinks=true, linktocpage=true, pdfstartpage=3, pdfstartview=FitV,%
  % uncomment the following line if you want to have black links (e.g., for printing)
  %colorlinks=false, linktocpage=false, pdfstartpage=3, pdfstartview=FitV, pdfborder={0 0 0},%
  breaklinks=true, pdfpagemode=UseNone, pageanchor=true, pdfpagemode=UseOutlines,%
  plainpages=false, bookmarksnumbered, bookmarksopen=true, bookmarksopenlevel=1,%
  hypertexnames=true, pdfhighlight=/O,%nesting=true,%frenchlinks,%
  urlcolor=baseRed, linkcolor=RoyalBlue, citecolor=RoyalBlue, %pagecolor=RoyalBlue,%
  %urlcolor=Black, linkcolor=Black, citecolor=Black, %pagecolor=Black,%
  pdftitle={\myTitle},%
  pdfauthor={\textcopyright\ \myName, \myUni, \myFaculty},%
  pdfsubject={},%
  pdfkeywords={},%
  pdfcreator={pdfLaTeX},%
  pdfproducer={LaTeX with hyperref and classicthesis}%
}

% ********************************************************************
% Setup autoreferences
% ********************************************************************
% There are some issues regarding autorefnames
% http://www.ureader.de/msg/136221647.aspx
% http://www.tex.ac.uk/cgi-bin/texfaq2html?label=latexwords
% you have to redefine the makros for the
% language you use, e.g., american, ngerman
% (as chosen when loading babel/AtBeginDocument)
% ********************************************************************
\makeatletter
\@ifpackageloaded{babel}%
  {%
    \addto\extrasamerican{%
      \renewcommand*{\figureautorefname}{Figure}%
      \renewcommand*{\tableautorefname}{Table}%
      \renewcommand*{\partautorefname}{Part}%
      \renewcommand*{\chapterautorefname}{Chapter}%
      \renewcommand*{\sectionautorefname}{Section}%
      \renewcommand*{\subsectionautorefname}{Section}%
      \renewcommand*{\subsubsectionautorefname}{Section}%
    }%
    \addto\extrasngerman{%
      \renewcommand*{\paragraphautorefname}{Absatz}%
      \renewcommand*{\subparagraphautorefname}{Unterabsatz}%
      \renewcommand*{\footnoteautorefname}{Fu\"snote}%
      \renewcommand*{\FancyVerbLineautorefname}{Zeile}%
      \renewcommand*{\theoremautorefname}{Theorem}%
      \renewcommand*{\appendixautorefname}{Anhang}%
      \renewcommand*{\equationautorefname}{Gleichung}%
      \renewcommand*{\itemautorefname}{Punkt}%
    }%
      % Fix to getting autorefs for subfigures right (thanks to Belinda Vogt for changing the definition)
      \providecommand{\subfigureautorefname}{\figureautorefname}%
    }{\relax}
\makeatother


% ****************************************************************************************************
% 7. Last calls before the bar closes
% ****************************************************************************************************
% ********************************************************************
% Development Stuff
% ********************************************************************
\listfiles
%\PassOptionsToPackage{l2tabu,orthodox,abort}{nag}
%  \usepackage{nag}
%\PassOptionsToPackage{warning, all}{onlyamsmath}
%  \usepackage{onlyamsmath}

% ********************************************************************
% Last, but not least...
% ********************************************************************
\usepackage{classicthesis}
% ****************************************************************************************************


% ****************************************************************************************************
% 8. Further adjustments (experimental)
% ****************************************************************************************************
% ********************************************************************
% Changing the text area
% ********************************************************************
%\areaset[current]{312pt}{761pt} % 686 (factor 2.2) + 33 head + 42 head \the\footskip
%\setlength{\marginparwidth}{7em}%
%\setlength{\marginparsep}{2em}%

% ********************************************************************
% Using different fonts
% ********************************************************************
% \usepackage[oldstylenums]{kpfonts} % oldstyle notextcomp
% \usepackage[osf]{libertine}
% \usepackage[light,condensed,math]{iwona}
% \renewcommand{\sfdefault}{iwona}
%\usepackage{lmodern} % <-- no osf support :-(
%\usepackage{cfr-lm} %
%\usepackage[urw-garamond]{mathdesign} <-- no osf support :-(
%\usepackage[default,osfigures]{opensans} % scale=0.95
%\usepackage[sfdefault]{FiraSans}
% ********************************************************************
% \usepackage[largesc,osf]{newpxtext}
% Used to fix these:
% https://bitbucket.org/amiede/classicthesis/issues/139/italics-in-pallatino-capitals-chapter
% https://bitbucket.org/amiede/classicthesis/issues/45/problema-testatine-su-classicthesis-style
% ********************************************************************
%\linespread{1.05} % a bit more for Palatino
% ****************************************************************************************************

% CUSTOM STUFF

% FOR CONFUSION MATRIX
\usepackage{multirow}
\usepackage{graphicx}
\newcommand\MyBox[1]{
  \fbox{\lower0.75cm
    \vbox to 1.8cm{\vfil
      \hbox to 1.8cm{\hfil\parbox{1.6cm}{#1}\hfil}
      \vfil}%
  }%
}

\usepackage{tabularcalc}
\usepackage{dirtree}

% FONT
\usepackage[default,light,bold]{sourceserifpro}
\usepackage[T1]{fontenc}

\titleformat{\section}
       {\fontsize{13}{17}\bfseries}{\thesection}{1em}{}
\titleformat{\subsection}
       {\fontsize{11}{17}\bfseries}{\thesubsection}{1em}{}
\titleformat{\subsubsection}
       {\fontsize{10}{17}\bfseries}{\thesubsubsection}{1em}{}

\usepackage{siunitx}
% \sisetup{detect-all}
\DeclareSIUnit\px{px}

\usepackage{cancel}

\usepackage{tikz}
\usepackage{pgfplots}
\pgfplotsset{compat=1.16}

% \usepackage[pdftex,active,tightpage]{preview} % vgg16
\usepackage{float}

\usepackage{enumerate}
\usepackage[inline, shortlabels]{enumitem}

\usepackage[cache=false, newfloat]{minted} % code
\newenvironment{code}{\captionsetup{type=listing}}{}
\SetupFloatingEnvironment{listing}{name=Listing}

\usepackage{blindtext,mdframed}

\usepackage{booktabs}
\newcommand{\tablestyle}[2]{\setlength{\tabcolsep}{#1}\renewcommand{\arraystretch}{#2}\centering\footnotesize}
\newlength\savewidth\newcommand\shline{\noalign{\global\savewidth\arrayrulewidth 
    \global\arrayrulewidth 1pt}\hline\noalign{\global\arrayrulewidth\savewidth}}   

\usepackage{makecell}
\usepackage{cellspace}
% \addparagraphcolumntypes{X}
\setlength\cellspacetoplimit{10pt}
\setlength\cellspacebottomlimit{10pt}

\usepackage[ngerman]{datetime}
\newdateformat{gerdate}{\THEDAY{ten }\monthnamengerman[\THEMONTH], \THEYEAR}

\newcommand{\TT}[1]{\texttt{#1}}
\newcommand{\BF}[1]{\textbf{#1}}
\newcommand{\IT}[1]{\textit{#1}}
\newcommand{\SC}[1]{\textsc{#1}}
\def\TODO{\textcolor{red}{\emph{TODO: }}}

\usepackage[ruled,vlined]{algorithm2e}
\usepackage[noend]{algpseudocode}

\usepackage{glossaries}
\renewcommand*\glspostdescription{\dotfill\nolinebreak}
\renewcommand*{\glstextformat}[1]{\slshape\color{RoyalBlue}{#1}}
\makeglossaries

\newglossaryentry{supervised}
{
    name=Supervised Learning,
    description={Das Lernen durch \textit{Pre-Labeled} Eingaben. Jedes Trainingsbeispiel besteht aus Eingabewerten $X$ und Zielwert $y$. Das Ziel ist es, den Klassifikationserror eines Modells zu reduzieren, sodass das Modell die korrekte Vorhersage $y$ trifft.}
}

\newglossaryentry{unsupervised}
{
    name=Unsupervised Learning,
    description={Unterscheidet sich vom \gls{supervised} dadurch, dass das Trainingsbeispiel keine Labels enthält. Dabei sollen selbstständig Muster in den Daten erkannt werden.}
}

\newglossaryentry{sliding}
{
    name=Sliding Window,
    description={In der Computer Vision ist damit ein \textit{Fenster} gemeint, welches über das gesamte Bild geschoben wird. Oft werden Fenster mit verschiedenen Größen und Seitenverhältnissen verwendet.}
}

\newglossaryentry{clustering}
{
    name=Clustering,
    description={Ist eine Methode des \gls{unsupervised}. Dabei werden Datenpunkte in $n$ Gruppen geteilt, sodass ähnliche Datenpunkte der gleichen Gruppe zugeordnet werden.}
}

\newglossaryentry{hysterese}
{
    name=Hysterese,
    description={Ist ein Systemverhalten, bei dem die Ausgabe abhängig von vorherigen Ausgaben ist. So können gleiche Eingaben zu unterschiedlichen Ausgaben führen.}
}

\setlength\parindent{0pt}
% ****************************************************************************************************
% If you like the classicthesis, then I would appreciate a postcard.
% My address can be found in the file ClassicThesis.pdf. A collection
% of the postcards I received so far is available online at
% http://postcards.miede.de
% ****************************************************************************************************


% ****************************************************************************************************
% 0. Set the encoding of your files. UTF-8 is the only sensible encoding nowadays. If you can't read
% äöüßáéçèê∂åëæƒÏ€ then change the encoding setting in your editor, not the line below. If your editor
% does not support utf8 use another editor!
% ****************************************************************************************************
\PassOptionsToPackage{utf8}{inputenc}
  \usepackage{inputenc}

% ****************************************************************************************************
% 1. Configure classicthesis for your needs here, e.g., remove "drafting" below
% in order to deactivate the time-stamp on the pages
% (see ClassicThesis.pdf for more information):
% ****************************************************************************************************
\PassOptionsToPackage{
  drafting=false,   % print version information on the bottom of the pages
  tocaligned=false, % the left column of the toc will be aligned (no indentation)
  dottedtoc=true,   % page numbers in ToC flushed right
  parts=true,       % use part division
  eulerchapternumbers=true, % use AMS Euler for chapter font (otherwise Palatino)
  linedheaders=false,       % chaper headers will have line above and beneath
  floatperchapter=true,     % numbering per chapter for all floats (i.e., Figure 1.1)
  listings=true,    % load listings package and setup LoL
  subfig=true,      % setup for preloaded subfig package
  eulermath=false,  % use awesome Euler fonts for mathematical formulae (only with pdfLaTeX)
  beramono=true,    % toggle a nice monospaced font (w/ bold)
  minionpro=false   % setup for minion pro font; use minion pro small caps as well (only with pdfLaTeX)
}{classicthesis}


% ****************************************************************************************************
% 2. Personal data and user ad-hoc commands
% ****************************************************************************************************
%\newcommand{\myTitle}{A Classic Thesis Style\xspace}
%\newcommand{\mySubtitle}{An Homage to The Elements of Typographic Style\xspace}
%\newcommand{\myDegree}{Doktor-Ingenieur (Dr.-Ing.)\xspace}
%\newcommand{\myName}{André Miede\xspace}
%\newcommand{\myProf}{Put name here\xspace}
%\newcommand{\myOtherProf}{Put name here\xspace}
%\newcommand{\mySupervisor}{Put name here\xspace}
%\newcommand{\myFaculty}{Put data here\xspace}
%\newcommand{\myDepartment}{Put data here\xspace}
%\newcommand{\myUni}{Put data here\xspace}
%\newcommand{\myLocation}{Saarbrücken\xspace}
%\newcommand{\myTime}{October 2017\xspace}
%\newcommand{\myVersion}{version 4.4}

% ********************************************************************
% Setup, finetuning, and useful commands
% ********************************************************************
\newcounter{dummy} % necessary for correct hyperlinks (to index, bib, etc.)
\newlength{\abcd} % for ab..z string length calculation
\providecommand{\mLyX}{L\kern-.1667em\lower.25em\hbox{Y}\kern-.125emX\@}
\newcommand{\ie}{i.\,e.}
\newcommand{\Ie}{I.\,e.}
\newcommand{\eg}{e.\,g.}
\newcommand{\Eg}{E.\,g.}
% ****************************************************************************************************


% ****************************************************************************************************
% 3. Loading some handy packages
% ****************************************************************************************************
% ********************************************************************
% Packages with options that might require adjustments
% ********************************************************************
%\PassOptionsToPackage{ngerman,american}{babel}   % change this to your language(s), main language last
% Spanish languages need extra options in order to work with this template
%\PassOptionsToPackage{spanish,es-lcroman}{babel}
\usepackage{babel}

\usepackage{csquotes}

\PassOptionsToPackage{%
  %backend=biber,bibencoding=utf8, %instead of bibtex
  backend=bibtex8,bibencoding=ascii,%
  language=auto,%
  style=numeric-comp,%
  %style=alphabetic,%
  %style=authoryear-comp, % Author 1999, 2010
  %bibstyle=authoryear,dashed=false, % dashed: substitute rep. author with ---
  % sorting=nyt, % name, year, title
  sorting=none, % sort by appearance
  maxbibnames=10, % default: 3, et al.
  %backref=true,%
  natbib=true % natbib compatibility mode (\citep and \citet still work)
}{biblatex}
  \usepackage{biblatex}

% \PassOptionsToPackage{fleqn}{amsmath}       % math environments and more by the AMS
  \usepackage{amsmath}

\PassOptionsToPackage{doublespacing}{hdathesis}  % options: abbrev exam big wiwi english master
  \usepackage{hdathesis}

% ********************************************************************
% General useful packages
% ********************************************************************
\PassOptionsToPackage{T1}{fontenc} % T2A for cyrillics
  \usepackage{fontenc}
\usepackage{textcomp} % fix warning with missing font shapes
\usepackage{scrhack} % fix warnings when using KOMA with listings package
\usepackage{xspace} % to get the spacing after macros right
\usepackage{mparhack} % get marginpar right
%\usepackage{fixltx2e} % fixes some LaTeX stuff --> since 2015 in the LaTeX kernel (see below)
% \usepackage[latest]{latexrelease} % emulate newer kernel version if older is detected
\PassOptionsToPackage{printonlyused}{acronym}
  \usepackage{acronym} % nice macros for handling all acronyms in the thesis
  %\renewcommand{\bflabel}[1]{{#1}\hfill} % fix the list of acronyms --> no longer working
  %\renewcommand*{\acsfont}[1]{\textsc{#1}}
  %\renewcommand*{\aclabelfont}[1]{\acsfont{#1}}
  %\def\bflabel#1{{#1\hfill}}
  \def\bflabel#1{{\acsfont{#1}\hfill}}
  % \def\aclabelfont#1{\acsfont{#1}}
% ****************************************************************************************************
% \usepackage{pgfplots} % External TikZ/PGF support (thanks to Andreas Nautsch) %\usetikzlibrary{external}
% \tikzexternalize[mode=list and make, prefix=ext-tikz/]
% \usepackage{tikz}
% ****************************************************************************************************


% ****************************************************************************************************
% 4. Setup floats: tables, (sub)figures, and captions
% ****************************************************************************************************
\usepackage{tabularx} % better tables
  \setlength{\extrarowheight}{3pt} % increase table row height
\newcommand{\tableheadline}[1]{\multicolumn{1}{c}{\spacedlowsmallcaps{#1}}}
\newcommand{\myfloatalign}{\centering} % to be used with each float for alignment
\usepackage{caption}
% Thanks to cgnieder and Claus Lahiri
% http://tex.stackexchange.com/questions/69349/spacedlowsmallcaps-in-caption-label
% [REMOVED DUE TO OTHER PROBLEMS, SEE ISSUE #82]
%\DeclareCaptionLabelFormat{smallcaps}{\bothIfFirst{#1}{~}\MakeTextLowercase{\textsc{#2}}}
%\captionsetup{font=small,labelformat=smallcaps} % format=hang,
\captionsetup{font=small, labelfont=bf} % format=hang,
\usepackage{subfig}
% ****************************************************************************************************


% ****************************************************************************************************
% 5. Setup code listings
% ****************************************************************************************************
\usepackage{listings}
%\lstset{emph={trueIndex,root},emphstyle=\color{BlueViolet}}%\underbar} % for special keywords
\lstset{language=[LaTeX]Tex,%C++,
  morekeywords={PassOptionsToPackage,selectlanguage},
  keywordstyle=\color{RoyalBlue},%\bfseries,
  basicstyle=\small\ttfamily,
  %identifierstyle=\color{NavyBlue},
  commentstyle=\color{Green}\ttfamily,
  stringstyle=\rmfamily,
  numbers=none,%left,%
  numberstyle=\scriptsize,%\tiny
  stepnumber=5,
  numbersep=8pt,
  showstringspaces=false,
  breaklines=true,
  %frameround=ftff,
  %frame=single,
  belowcaptionskip=.75\baselineskip
  %frame=L
}
% ****************************************************************************************************


% ****************************************************************************************************
% 6. PDFLaTeX, hyperreferences, and citation backreferences
% ****************************************************************************************************
% ********************************************************************
% Using PDFLaTeX
% ********************************************************************
\PassOptionsToPackage{hyperfootnotes=false,pdfpagelabels}{hyperref}
  \usepackage{hyperref}  % backref linktocpage pagebackref
%\ifpdf
%\pdfcompresslevel=9
%\pdfadjustspacing=1
%\fi
%\PassOptionsToPackage{pdftex}{graphicx} %%%IVO: driver will be chosen automatically
  \usepackage{graphicx}


% ********************************************************************
% Hyperreferences
% ********************************************************************
\hypersetup{%
  %draft, % hyperref's draft mode, for printing see below
  colorlinks=true, linktocpage=true, pdfstartpage=3, pdfstartview=FitV,%
  % uncomment the following line if you want to have black links (e.g., for printing)
  %colorlinks=false, linktocpage=false, pdfstartpage=3, pdfstartview=FitV, pdfborder={0 0 0},%
  breaklinks=true, pdfpagemode=UseNone, pageanchor=true, pdfpagemode=UseOutlines,%
  plainpages=false, bookmarksnumbered, bookmarksopen=true, bookmarksopenlevel=1,%
  hypertexnames=true, pdfhighlight=/O,%nesting=true,%frenchlinks,%
  urlcolor=baseRed, linkcolor=RoyalBlue, citecolor=RoyalBlue, %pagecolor=RoyalBlue,%
  %urlcolor=Black, linkcolor=Black, citecolor=Black, %pagecolor=Black,%
  pdftitle={\myTitle},%
  pdfauthor={\textcopyright\ \myName, \myUni, \myFaculty},%
  pdfsubject={},%
  pdfkeywords={},%
  pdfcreator={pdfLaTeX},%
  pdfproducer={LaTeX with hyperref and classicthesis}%
}

% ********************************************************************
% Setup autoreferences
% ********************************************************************
% There are some issues regarding autorefnames
% http://www.ureader.de/msg/136221647.aspx
% http://www.tex.ac.uk/cgi-bin/texfaq2html?label=latexwords
% you have to redefine the makros for the
% language you use, e.g., american, ngerman
% (as chosen when loading babel/AtBeginDocument)
% ********************************************************************
\makeatletter
\@ifpackageloaded{babel}%
  {%
    \addto\extrasamerican{%
      \renewcommand*{\figureautorefname}{Figure}%
      \renewcommand*{\tableautorefname}{Table}%
      \renewcommand*{\partautorefname}{Part}%
      \renewcommand*{\chapterautorefname}{Chapter}%
      \renewcommand*{\sectionautorefname}{Section}%
      \renewcommand*{\subsectionautorefname}{Section}%
      \renewcommand*{\subsubsectionautorefname}{Section}%
    }%
    \addto\extrasngerman{%
      \renewcommand*{\paragraphautorefname}{Absatz}%
      \renewcommand*{\subparagraphautorefname}{Unterabsatz}%
      \renewcommand*{\footnoteautorefname}{Fu\"snote}%
      \renewcommand*{\FancyVerbLineautorefname}{Zeile}%
      \renewcommand*{\theoremautorefname}{Theorem}%
      \renewcommand*{\appendixautorefname}{Anhang}%
      \renewcommand*{\equationautorefname}{Gleichung}%
      \renewcommand*{\itemautorefname}{Punkt}%
    }%
      % Fix to getting autorefs for subfigures right (thanks to Belinda Vogt for changing the definition)
      \providecommand{\subfigureautorefname}{\figureautorefname}%
    }{\relax}
\makeatother


% ****************************************************************************************************
% 7. Last calls before the bar closes
% ****************************************************************************************************
% ********************************************************************
% Development Stuff
% ********************************************************************
\listfiles
%\PassOptionsToPackage{l2tabu,orthodox,abort}{nag}
%  \usepackage{nag}
%\PassOptionsToPackage{warning, all}{onlyamsmath}
%  \usepackage{onlyamsmath}

% ********************************************************************
% Last, but not least...
% ********************************************************************
\usepackage{classicthesis}
% ****************************************************************************************************


% ****************************************************************************************************
% 8. Further adjustments (experimental)
% ****************************************************************************************************
% ********************************************************************
% Changing the text area
% ********************************************************************
%\areaset[current]{312pt}{761pt} % 686 (factor 2.2) + 33 head + 42 head \the\footskip
%\setlength{\marginparwidth}{7em}%
%\setlength{\marginparsep}{2em}%

% ********************************************************************
% Using different fonts
% ********************************************************************
% \usepackage[oldstylenums]{kpfonts} % oldstyle notextcomp
% \usepackage[osf]{libertine}
% \usepackage[light,condensed,math]{iwona}
% \renewcommand{\sfdefault}{iwona}
%\usepackage{lmodern} % <-- no osf support :-(
%\usepackage{cfr-lm} %
%\usepackage[urw-garamond]{mathdesign} <-- no osf support :-(
%\usepackage[default,osfigures]{opensans} % scale=0.95
%\usepackage[sfdefault]{FiraSans}
% ********************************************************************
% \usepackage[largesc,osf]{newpxtext}
% Used to fix these:
% https://bitbucket.org/amiede/classicthesis/issues/139/italics-in-pallatino-capitals-chapter
% https://bitbucket.org/amiede/classicthesis/issues/45/problema-testatine-su-classicthesis-style
% ********************************************************************
%\linespread{1.05} % a bit more for Palatino
% ****************************************************************************************************

% CUSTOM STUFF

% FOR CONFUSION MATRIX
\usepackage{multirow}
\usepackage{graphicx}
\newcommand\MyBox[1]{
  \fbox{\lower0.75cm
    \vbox to 1.8cm{\vfil
      \hbox to 1.8cm{\hfil\parbox{1.6cm}{#1}\hfil}
      \vfil}%
  }%
}

\usepackage{tabularcalc}
\usepackage{dirtree}

% FONT
\usepackage[default,light,bold]{sourceserifpro}
\usepackage[T1]{fontenc}

\titleformat{\section}
       {\fontsize{13}{17}\bfseries}{\thesection}{1em}{}
\titleformat{\subsection}
       {\fontsize{11}{17}\bfseries}{\thesubsection}{1em}{}
\titleformat{\subsubsection}
       {\fontsize{10}{17}\bfseries}{\thesubsubsection}{1em}{}

\usepackage{siunitx}
% \sisetup{detect-all}
\DeclareSIUnit\px{px}

\usepackage{cancel}

\usepackage{tikz}
\usepackage{pgfplots}
\pgfplotsset{compat=1.16}

% \usepackage[pdftex,active,tightpage]{preview} % vgg16
\usepackage{float}

\usepackage{enumerate}
\usepackage[inline, shortlabels]{enumitem}

\usepackage[cache=false, newfloat]{minted} % code
\newenvironment{code}{\captionsetup{type=listing}}{}
\SetupFloatingEnvironment{listing}{name=Listing}

\usepackage{blindtext,mdframed}

\usepackage{booktabs}
\newcommand{\tablestyle}[2]{\setlength{\tabcolsep}{#1}\renewcommand{\arraystretch}{#2}\centering\footnotesize}
\newlength\savewidth\newcommand\shline{\noalign{\global\savewidth\arrayrulewidth 
    \global\arrayrulewidth 1pt}\hline\noalign{\global\arrayrulewidth\savewidth}}   

\usepackage{makecell}
\usepackage{cellspace}
% \addparagraphcolumntypes{X}
\setlength\cellspacetoplimit{10pt}
\setlength\cellspacebottomlimit{10pt}

\usepackage[ngerman]{datetime}
\newdateformat{gerdate}{\THEDAY{ten }\monthnamengerman[\THEMONTH], \THEYEAR}

\newcommand{\TT}[1]{\texttt{#1}}
\newcommand{\BF}[1]{\textbf{#1}}
\newcommand{\IT}[1]{\textit{#1}}
\newcommand{\SC}[1]{\textsc{#1}}
\def\TODO{\textcolor{red}{\emph{TODO: }}}

\usepackage[ruled,vlined]{algorithm2e}
\usepackage[noend]{algpseudocode}

\usepackage{glossaries}
\renewcommand*\glspostdescription{\dotfill\nolinebreak}
\renewcommand*{\glstextformat}[1]{\slshape\color{RoyalBlue}{#1}}
\makeglossaries

\newglossaryentry{supervised}
{
    name=Supervised Learning,
    description={Das Lernen durch \textit{Pre-Labeled} Eingaben. Jedes Trainingsbeispiel besteht aus Eingabewerten $X$ und Zielwert $y$. Das Ziel ist es, den Klassifikationserror eines Modells zu reduzieren, sodass das Modell die korrekte Vorhersage $y$ trifft.}
}

\newglossaryentry{unsupervised}
{
    name=Unsupervised Learning,
    description={Unterscheidet sich vom \gls{supervised} dadurch, dass das Trainingsbeispiel keine Labels enthält. Dabei sollen selbstständig Muster in den Daten erkannt werden.}
}

\newglossaryentry{sliding}
{
    name=Sliding Window,
    description={In der Computer Vision ist damit ein \textit{Fenster} gemeint, welches über das gesamte Bild geschoben wird. Oft werden Fenster mit verschiedenen Größen und Seitenverhältnissen verwendet.}
}

\newglossaryentry{clustering}
{
    name=Clustering,
    description={Ist eine Methode des \gls{unsupervised}. Dabei werden Datenpunkte in $n$ Gruppen geteilt, sodass ähnliche Datenpunkte der gleichen Gruppe zugeordnet werden.}
}

\newglossaryentry{hysterese}
{
    name=Hysterese,
    description={Ist ein Systemverhalten, bei dem die Ausgabe abhängig von vorherigen Ausgaben ist. So können gleiche Eingaben zu unterschiedlichen Ausgaben führen.}
}

\setlength\parindent{0pt}